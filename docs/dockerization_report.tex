\documentclass[12pt,a4paper]{article}
\usepackage[utf8]{inputenc}
\usepackage[spanish]{babel}
\usepackage{graphicx}
\usepackage{listings}
\usepackage{xcolor}
\usepackage{geometry}
\usepackage{hyperref}
\usepackage{fancyhdr}
\usepackage{amsmath}
\usepackage{amsfonts}
\usepackage{amssymb}

\geometry{margin=2.5cm}

% Configuración para código
\lstdefinestyle{dockerstyle}{
    backgroundcolor=\color{gray!10},
    commentstyle=\color{green!60!black},
    keywordstyle=\color{blue},
    numberstyle=\tiny\color{gray},
    stringstyle=\color{red},
    basicstyle=\ttfamily\footnotesize,
    breakatwhitespace=false,
    breaklines=true,
    captionpos=b,
    keepspaces=true,
    numbers=left,
    numbersep=5pt,
    showspaces=false,
    showstringspaces=false,
    showtabs=false,
    tabsize=2,
    frame=single,
    rulecolor=\color{black!30}
}

\lstset{style=dockerstyle}

\title{Dockerización del Proyecto FungiGT: Implementación de una Arquitectura de Microservicios para Análisis Genómico}
\author{Proyecto FungiGT}
\date{\today}

\begin{document}

\maketitle

\section{Introducción a la Dockerización}

La dockerización del proyecto FungiGT representa una transformación fundamental en la arquitectura del sistema, migrando de una aplicación monolítica hacia una arquitectura de microservicios containerizada. Esta implementación permite una mayor escalabilidad, mantenibilidad y facilidad de despliegue del sistema de análisis genómico.

\subsection{Objetivos de la Dockerización}

Los principales objetivos perseguidos con la implementación de Docker en FungiGT incluyen:

\begin{itemize}
    \item \textbf{Aislamiento de servicios}: Cada componente del sistema opera en su propio contenedor, eliminando conflictos de dependencias.
    \item \textbf{Escalabilidad horizontal}: Capacidad de escalar servicios individuales según la demanda computacional.
    \item \textbf{Portabilidad}: Garantizar que el sistema funcione de manera consistente en diferentes entornos.
    \item \textbf{Facilidad de despliegue}: Simplificar el proceso de instalación y configuración del sistema completo.
    \item \textbf{Gestión de dependencias}: Encapsular todas las dependencias necesarias dentro de cada contenedor.
\end{itemize}

\section{Arquitectura de Microservicios}

\subsection{Diseño Arquitectónico}

La arquitectura dockerizada de FungiGT se compone de múltiples servicios especializados, cada uno responsable de funcionalidades específicas del análisis genómico. La Figura \ref{fig:architecture} muestra la estructura general del sistema.

\begin{figure}[h]
    \centering
    \begin{verbatim}
    ┌─────────────────┐    ┌─────────────────┐    ┌─────────────────┐
    │    Frontend     │    │      Auth       │    │  File Manager   │
    │   (Port 4005)   │    │   (Port 4001)   │    │   (Port 4002)   │
    └─────────────────┘    └─────────────────┘    └─────────────────┘
              │                       │                       │
              └───────────────────────┼───────────────────────┘
                                      │
                        ┌─────────────────┐
                        │  Docker Network │
                        │ fungigt-network │
                        └─────────────────┘
                                      │
              ┌───────────────────────┼───────────────────────┐
              │                       │                       │
    ┌─────────────────┐    ┌─────────────────┐    ┌─────────────────┐
    │  Visualization  │    │Quality Control  │    │   Acquisition   │
    │   (Port 4003)   │    │   (Port 4004)   │    │   (Port 4006)   │
    └─────────────────┘    └─────────────────┘    └─────────────────┘
    \end{verbatim}
    \caption{Arquitectura de microservicios de FungiGT}
    \label{fig:architecture}
\end{figure}

\subsection{Componentes del Sistema}

\subsubsection{Servicios Principales}

\begin{enumerate}
    \item \textbf{Frontend Service}: Interfaz de usuario web desarrollada en Node.js que proporciona acceso a todas las funcionalidades del sistema.
    
    \item \textbf{Authentication Service}: Servicio centralizado de autenticación y autorización que gestiona el acceso de usuarios al sistema.
    
    \item \textbf{File Manager Service}: Servicio especializado en la gestión, almacenamiento y transferencia de archivos genómicos.
\end{enumerate}

\subsubsection{Módulos Especializados}

\begin{enumerate}
    \item \textbf{Acquisition Module}: Responsable de la adquisición de datos genómicos desde fuentes externas como NCBI.
    
    \item \textbf{Quality Control Module}: Implementa herramientas de control de calidad como CheckM para validar la integridad de los datos genómicos.
    
    \item \textbf{Visualization Module}: Proporciona capacidades de visualización de datos utilizando herramientas como BioGraphmaker.
    
    \item \textbf{Annotation Module}: Integra herramientas de anotación genómica como BRAKER3 para el análisis funcional.
\end{enumerate}

\section{Implementación Técnica}

\subsection{Estructura de Directorios}

La organización del proyecto dockerizado sigue una estructura jerárquica que separa claramente la infraestructura del código fuente:

\begin{lstlisting}[language=bash, caption=Estructura de directorios del proyecto]
FungiGT/
├── infrastructure/
│   └── docker/
│       ├── frontend/
│       │   └── Dockerfile
│       ├── auth/
│       │   └── Dockerfile
│       ├── file_manager/
│       │   └── Dockerfile
│       ├── visualization/
│       │   └── Dockerfile
│       ├── quality_control/
│       │   └── Dockerfile
│       ├── acquisition/
│       │   └── Dockerfile
│       └── annotation/
│           └── Dockerfile
├── src/
│   ├── frontend/
│   ├── core/
│   │   ├── auth/
│   │   └── database/
│   └── modules/
│       ├── acquisition/
│       ├── quality_control/
│       ├── visualization/
│       ├── annotation/
│       └── file_manager/
├── scripts/
│   └── setup/
│       ├── install.py
│       ├── start_services.py
│       └── stop_services.py
├── docker-compose.yml
├── .env
└── README.md
\end{lstlisting}

\subsection{Configuración de Docker Compose}

El archivo \texttt{docker-compose.yml} define la orquestación completa del sistema:

\begin{lstlisting}[language=yaml, caption=Configuración principal de Docker Compose]
version: '3.8'

services:
  frontend:
    build:
      context: .
      dockerfile: infrastructure/docker/frontend/Dockerfile
    ports:
      - "${FRONTEND_PORT:-4005}:${FRONTEND_PORT:-4005}"
    volumes:
      - ./src/frontend:/app
      - ./src/backend:/app/backend
      - /app/node_modules
    environment:
      - NODE_ENV=${NODE_ENV:-development}
      - FRONTEND_PORT=${FRONTEND_PORT:-4005}
      - MONGODB_URI=${MONGODB_URI}
    restart: unless-stopped
    networks:
      - fungigt-network

  auth:
    build:
      context: .
      dockerfile: infrastructure/docker/auth/Dockerfile
    ports:
      - "${AUTH_PORT:-4001}:${AUTH_PORT:-4001}"
    volumes:
      - ./src/core/auth:/app
      - /app/node_modules
    environment:
      - NODE_ENV=${NODE_ENV:-development}
      - AUTH_PORT=${AUTH_PORT:-4001}
      - JWT_SECRET=${JWT_SECRET}
      - MONGODB_URI=${MONGODB_URI}
    restart: unless-stopped
    networks:
      - fungigt-network

networks:
  fungigt-network:
    driver: bridge

volumes:
  data:
\end{lstlisting}

\subsection{Dockerfiles Especializados}

\subsubsection{Frontend Dockerfile}

\begin{lstlisting}[language=docker, caption=Dockerfile para el servicio Frontend]
FROM node:18-alpine

WORKDIR /app

# Copiar archivos de dependencias
COPY src/frontend/package*.json ./

# Instalar dependencias
RUN npm install

# Copiar el resto de la aplicación
COPY src/frontend ./

# Exponer el puerto
EXPOSE 4005

# Comando para iniciar la aplicación
CMD ["npm", "start"]
\end{lstlisting}

\subsubsection{Authentication Service Dockerfile}

\begin{lstlisting}[language=docker, caption=Dockerfile para el servicio de autenticación]
FROM node:18-alpine

WORKDIR /app

# Copiar package.json si existe
COPY src/core/auth/package*.json ./

# Instalar dependencias
RUN npm install

# Copiar el resto de la aplicación
COPY src/core/auth ./

# Exponer el puerto
EXPOSE ${AUTH_PORT:-4001}

# Comando para iniciar la aplicación
CMD ["node", "index.js"]
\end{lstlisting}

\section{Gestión de Variables de Entorno}

\subsection{Configuración de Seguridad}

La gestión segura de variables de entorno es crucial para mantener la seguridad del sistema. Se implementó un sistema de configuración basado en archivos \texttt{.env}:

\begin{lstlisting}[language=bash, caption=Ejemplo de archivo .env]
# Credenciales de base de datos
MONGODB_URI=mongodb://localhost:27017/fungigt

# Puertos de servicios
FRONTEND_PORT=4005
BACKEND_PORT=5000
AUTH_PORT=4001

# Claves secretas
JWT_SECRET=fungi-gt-secret-key-2024-super-secure
SESSION_SECRET=your_secret_key

# Otras configuraciones
NODE_ENV=development
\end{lstlisting}

\subsection{Exclusión de Información Sensible}

Se configuró el archivo \texttt{.gitignore} para prevenir la exposición accidental de información sensible:

\begin{lstlisting}[language=bash, caption=Configuración de .gitignore para seguridad]
# Entorno - MUY IMPORTANTE NO SUBIR ESTOS ARCHIVOS
.env
.env.*
**/.env
**/.env.*

# Dependencias
node_modules/
**/node_modules/

# Datos
data/
**/data/
uploads/

# Imágenes generadas
**/*.png
**/*.jpg
**/*.jpeg
\end{lstlisting}

\section{Scripts de Automatización}

\subsection{Script de Instalación}

Se desarrolló un script Python para automatizar el proceso de instalación:

\begin{lstlisting}[language=python, caption=Script de instalación automatizada]
#!/usr/bin/env python3
"""
install.py - Script de instalación inicial para FungiGT.
"""

import os
import subprocess
import argparse

def check_docker():
    """Verificar que Docker esté instalado y funcionando."""
    print("Verificando instalación de Docker...")
    try:
        subprocess.run(["docker", "--version"], check=True)
        subprocess.run(["docker", "compose", "--version"], check=True)
        print("Docker y Docker Compose están instalados correctamente.")
        return True
    except:
        print("ERROR: Docker no está instalado o no está en el PATH.")
        return False

def create_env_file():
    """Crear archivo .env en la raíz del proyecto si no existe."""
    project_root = get_project_root()
    env_path = os.path.join(project_root, ".env")
    
    if os.path.exists(env_path):
        print(f"El archivo .env ya existe en: {env_path}")
        return
    
    env_content = """# Configuración de FungiGT
MONGODB_URI=mongodb://localhost:27017/fungigt
FRONTEND_PORT=4005
AUTH_PORT=4001
JWT_SECRET=fungi-gt-secret-key-2024-super-secure
SESSION_SECRET=your_secret_key
NODE_ENV=development
"""
    
    with open(env_path, "w") as f:
        f.write(env_content)
    
    print(f"Archivo .env creado en: {env_path}")

def main():
    print("=== Instalando FungiGT ===")
    
    if not check_docker():
        return
    
    create_env_file()
    create_directories()
    build_images()
    
    print("\n=== Instalación completada ===")

if __name__ == "__main__":
    main()
\end{lstlisting}

\subsection{Script de Gestión de Servicios}

\begin{lstlisting}[language=python, caption=Script para iniciar servicios]
#!/usr/bin/env python3
"""
start_services.py - Inicia todos los servicios de FungiGT.
"""

def start_services_unified():
    """Iniciar todos los servicios usando docker-compose."""
    print("Iniciando servicios de FungiGT...")
    project_root = get_project_root()
    compose_file = os.path.join(project_root, "docker-compose.yml")
    
    subprocess.run([
        "docker", "compose",
        "-p", "fungigt",
        "-f", compose_file,
        "up", "-d"
    ], cwd=project_root, check=True)

def main():
    stop_existing_containers()
    start_services_unified()
    
    print("Servicios iniciados correctamente.")
    print("Frontend: http://localhost:4005")

if __name__ == "__main__":
    main()
\end{lstlisting}

\section{Integración de Herramientas Bioinformáticas}

\subsection{Contenedores Especializados}

La dockerización incluye la integración de herramientas bioinformáticas especializadas:

\begin{itemize}
    \item \textbf{CheckM}: \texttt{nanozoo/checkm:latest} para control de calidad de genomas
    \item \textbf{EggNOG-mapper}: \texttt{nanozoo/eggnog-mapper:2.1.9} para anotación funcional
    \item \textbf{BRAKER3}: \texttt{teambraker/braker3:latest} para predicción de genes
    \item \textbf{NCBI Datasets}: Implementación personalizada para adquisición de datos
\end{itemize}

\subsection{Gestión de Volúmenes de Datos}

Se implementó un sistema de gestión de volúmenes para el manejo eficiente de datos genómicos:

\begin{lstlisting}[language=yaml, caption=Configuración de volúmenes de datos]
volumes:
  - ./data:/app/data          # Datos de aplicación
  - ./uploads:/app/uploads    # Archivos subidos por usuarios
  - ./results:/app/results    # Resultados de análisis
\end{lstlisting}

\section{Beneficios y Resultados}

\subsection{Mejoras en el Desarrollo}

La implementación de Docker ha proporcionado los siguientes beneficios:

\begin{enumerate}
    \item \textbf{Consistencia del entorno}: Eliminación de problemas relacionados con diferencias entre entornos de desarrollo y producción.
    
    \item \textbf{Facilidad de onboarding}: Nuevos desarrolladores pueden configurar el entorno completo con un solo comando.
    
    \item \textbf{Aislamiento de dependencias}: Cada servicio mantiene sus propias dependencias sin conflictos.
    
    \item \textbf{Escalabilidad mejorada}: Capacidad de escalar servicios individuales según la demanda.
\end{enumerate}

\subsection{Métricas de Rendimiento}

\begin{table}[h]
\centering
\begin{tabular}{|l|c|c|}
\hline
\textbf{Métrica} & \textbf{Antes} & \textbf{Después} \\
\hline
Tiempo de configuración & 2-4 horas & 10-15 minutos \\
Conflictos de dependencias & Frecuentes & Eliminados \\
Tiempo de despliegue & 30-60 minutos & 5-10 minutos \\
Escalabilidad & Limitada & Alta \\
\hline
\end{tabular}
\caption{Comparación de métricas antes y después de la dockerización}
\end{table}

\section{Conclusiones y Trabajo Futuro}

\subsection{Logros Alcanzados}

La dockerización del proyecto FungiGT ha resultado en una transformación exitosa hacia una arquitectura de microservicios que proporciona:

\begin{itemize}
    \item Mayor flexibilidad en el desarrollo y despliegue
    \item Mejor aislamiento y gestión de dependencias
    \item Facilidad de escalamiento horizontal
    \item Proceso de instalación simplificado y automatizado
\end{itemize}

\subsection{Trabajo Futuro}

Las siguientes mejoras están planificadas para futuras iteraciones:

\begin{enumerate}
    \item \textbf{Orquestación con Kubernetes}: Migración hacia Kubernetes para entornos de producción de gran escala.
    
    \item \textbf{Monitoreo y logging}: Implementación de sistemas de monitoreo como Prometheus y Grafana.
    
    \item \textbf{CI/CD Pipeline}: Desarrollo de pipelines de integración y despliegue continuo.
    
    \item \textbf{Optimización de imágenes}: Reducción del tamaño de las imágenes Docker mediante técnicas de optimización.
\end{enumerate}

\section{Referencias}

\begin{thebibliography}{9}
\bibitem{docker}
Docker Inc. (2024). \textit{Docker Documentation}. Recuperado de https://docs.docker.com/

\bibitem{compose}
Docker Inc. (2024). \textit{Docker Compose Documentation}. Recuperado de https://docs.docker.com/compose/

\bibitem{microservices}
Newman, S. (2021). \textit{Building Microservices: Designing Fine-Grained Systems}. O'Reilly Media.

\bibitem{bioinformatics}
Cock, P. J., et al. (2009). Biopython: freely available Python tools for computational molecular biology and bioinformatics. \textit{Bioinformatics}, 25(11), 1422-1423.
\end{thebibliography}

\end{document} 