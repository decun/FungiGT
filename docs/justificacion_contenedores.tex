\subsection{Justificación del Uso de Contenedores}

La implementación de contenedores Docker en el proyecto FungiGT se justifica por la naturaleza heterogénea y compleja de las herramientas bioinformáticas utilizadas. El análisis genómico requiere la integración de múltiples software especializados como CheckM, EggNOG-mapper, BRAKER3 y NCBI Datasets, cada uno con dependencias específicas y potencialmente conflictivas en términos de versiones de librerías, intérpretes de Python, y configuraciones del sistema operativo. Los contenedores proporcionan un entorno aislado y reproducible que encapsula cada herramienta con sus dependencias exactas, eliminando los problemas de compatibilidad que tradicionalmente plagaban los pipelines bioinformáticos. Además, la naturaleza computacionalmente intensiva de los análisis genómicos demanda escalabilidad horizontal, donde diferentes módulos pueden requerir recursos variables según el tamaño de los datasets y la complejidad de los análisis. La containerización permite escalar servicios individuales de manera independiente, optimizando el uso de recursos computacionales y facilitando el despliegue en infraestructuras distribuidas, desde estaciones de trabajo locales hasta clusters de alto rendimiento en la nube. 